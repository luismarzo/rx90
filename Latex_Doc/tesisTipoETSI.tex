%:Clase del documento
\documentclass[fontsize=10pt, Myfinal=true, twoside, numbers=noenddot]{scrbook}
%, Myfinal=true, Minion=true, English=true

%:Paquete de estilos propuesto
\usepackage{libroETSI}

%:Paquete específico para cargar tikz (y sus librerías) y pgfplots
\usepackage{dtsc-creafig}

%:Paquete para notaciones específicas
\usepackage{notacion}

%:Paquete para incorporar aspectos concretos de la edición
\usepackage{edicionTesis}

%:Estas líneas de código son INNECESARIAS excepto para mostrar determinadas características en este manual. Pueden eliminarse o comentarse sin ningún problema.
%Se usan para compilar el capítulo estilolibroetsi.tex
\usepackage[final]{showexpl}
\lstset{explpreset={frame=none,rframe={}, numbers=none,numbersep=3pt, columns=flexible,language={[LaTeX]TeX},basicstyle=\ttfamily,keywordstyle=\color{blue}}}%numberstyle=\tiny,

\makeatletter
\patchcmd{\SX@codeInput}{xleftmargin=0pt,xrightmargin=0pt}{}
  {\typeout{***Successfully patched \protect\SX@codeInput***}}
  {\typeout{***ERROR! Failed to patch \protect\SX@codeInput***}}
\makeatother
%:Hasta AQUI

%:Para modificar fácilmente la fuente del texto. 
\makeatletter
\ifdtsc@Minion % Queremos utilizar la fuente Minion y lo hemos declarado al principio
	\ifluatex
		\setmainfont[Renderer=Basic, Ligatures=TeX,	% Fuente del texto 
		Scale=1.01,
		]{Minion Pro}
   		% En este caso conviene modificar ligeramente el tamaño de las fuentes matemáticas
		\DeclareMathSizes{10}{10.5}{7.35}{5.25}
		\DeclareMathSizes{10.95}{11.55}{8.08}{5.77}
		\DeclareMathSizes{12}{12.6}{8.82}{6.3}
%		\setmainfont[Renderer=Basic, Ligatures=TeX,	% Fuente del texto 
%		]{Adobe Garamond Pro}
%		\setmainfont[Renderer=Basic, Ligatures=TeX,	% Fuente del texto 
%		]{Palatino LT Std}
	\fi
\else
	\ifluatex
		% Para utilizar la fuente Times New Roman, o alguna otra que se tenga instalada
		\setmainfont[Renderer=Basic, Ligatures=TeX,	% Fuente del texto 
		Scale=1.0,
		]{Times New Roman}
	\else
		\usepackage{tgtermes} 	%clone of Times
		%\usepackage[default]{droidserif}
		%\usepackage{anttor} 	
	\fi
\fi
\makeatother

%Por si quieren usar bibliografía con BIBER
%BIBER%%:Para la bibliografía en BIBER, descomentar las líneas siguientes
%\defbibheading{etsi}[]{%
%	\chapter*{Bibliografía}%
%	\chaptermark{Bibliografía} 
%	\markboth{#1}{#1}}
%\addbibresource{bibliografiaLibroETSI.bib}

% Ejemplo de Glosario
\newacronym[type=main]{ETSI}{ETSI}{Escuela Técnica Superior de Ingeniería}
\newacronym[type=main]{US}{US}{Universidad de Sevilla}
\newacronym[type=main]{DMC}{DMC}{Canal Discreto sin Memoria}

\makeindex
\makeglossaries %Si no se quiere el glosario, comentar esta línea.

%TAMAÑO LIBRO O A4
%Para definir el tamaño del documento, hay que elegir uno de los siguientes y comentar el otro
%Formato Libro
\geometry
{paperheight=240mm,%
paperwidth=170mm,%
top=25mm,%
headsep=7.5mm,%
footskip=10mm,%
textheight=190mm,%
textwidth=124mm,%
bindingoffset=15mm,%
twoside}

%\usepackage[a4,center, cross]{crop}%para poner las cruces de esquina de página, poner la opción cross
% Formato A4
%\usepackage[paperheight=297mm, paperwidth=210mm, top=25mm, headsep=8.5mm, includefoot, textheight=240mm, textwidth=150mm, bindingoffset=0mm, twoside]{geometry}

%:Esquema de numeración por defecto
\setenumerate[1]{label=\normalfont\bfseries{\arabic*.}, leftmargin=*, labelindent=\parindent}
\setenumerate[2]{label=\normalfont\bfseries{\alph*}), leftmargin=*}
\setenumerate[3]{label=\normalfont\bfseries{\roman*.}, leftmargin=*}
\setlist{itemsep=.1em}
\setlength{\parindent}{1.0 em}

\setcounter{tocdepth}{4}						% El nivel hasta el que se muestra el índice 

%:Empieza el documento
\begin{document}


%:Construcción de la portada y hojas obligatorias. Ver edicionTesis.sty para posibles modificaciones
\titulo{Formato de Publicación de la Escuela Técnica Superior de Ingeniería}
\autor{F. Javier Payán Somet}
\director{Juan José Murillo Fuentes}
\titulodirector{Profesor Titular}

\departamento{Teoría de la Señal y  Comunicaciones}
\centro{Escuela Técnica Superior de Ingeniería}
\universidad{Universidad de Sevilla}
\titulacion{Ingeniería de Telecomunicación}
\fecha{2014}
\nombretrabajo{Tesis Doctoral} %Trabajo Fin de Grado, Proyecto fin de Máster,....


\portadaTesis{figuras/imagenlibro.png}


%:Para establecer características del pdf generado
\hypersetup
	{
 	linkcolor=black, %Tocar para poner color en enlaces
	pdfauthor={\elautor},
	pdftitle={\eltitulo}, 
	pdfkeywords={Formato de Tesis, Latex, ETSI, Universidad de Sevilla}	
	 }


%:Todo lo que constituye la primera parte del texto que no es el cuerpo del texto en realidad
\frontmatter
\pagenumbering{Roman} %Pone la numeración en mayúscula (En español parece que es obligatorio)

\dedicatoria{A mi familia\\A mis profesores} 

\include{agradecimientos/agradecimientos}

\include{resumen/resumen}

% Índice abreviado 
% El índice abreviado se incluye también en algunos textos, con menor detalle que el completo. Descomentar las siguientes líneas.
\cleardoublepage
\phantomsection
\addcontentsline{toc}{listasf}{\shortcontentsname}
\pagestyle{especial}
\shorttoc{\shortcontentsname}{1}

%Índice normal, el completo
\cleardoublepage
\phantomsection
\pagestyle{especial}
\tableofcontents

\include{notacion/notacion} %No incluir si no se quiere, comentándolo

%:Empieza el contenido del libro
\mainmatter

%:Para incluir toda la referencia bibliográfica aunque no se cite, descomente la siguiente línea
%\nocite{*}

%:Página por defecto
\pagestyle{esitscCD}

%:Los diferentes capítulos, en carpetas separadas
\include{introduccion/introduccion}

\include{CapituloLibroETSI/CapituloLibroETSI} 

\include{CapituloProblemasLibroETSI/CapituloProblemasETSI} 
 
\include{CapituloEstilo/estilolibroetsi}

\include{CapituloEstilo/estilolibroetsi} 
%:Empezamos con los apéndices, que irían en uno o más ficheros. Es necesario incluir estos ficheros entre el entorno \begin{appendices}....\end{appendices} debido a que se ha deseado utilizar un formato diferente para el título de los apéndices, incluyendo la palabra apéndice, para la numeración de los apéndices, alfabético, y para las cabeceras de las páginas.

\begin{appendices}

% Fichero en el que se incluyen los apéndices
\include{apendices/apendices} %Ver este fichero para incluir ahí los apéndices.

\end{appendices}
%:Fin de la inclusión de apéndices

%:Empieza todo lo que no constituye el cuerpo en si del libro. Todo lo que va detrás
\backmatter

%:Indice de figuras, coméntese las siguientes líneas si no se desea
\cleardoublepage
\phantomsection

%:Para añadir una línea en blanco en el TOC y separar esta lista
\addtocontents{toc}{\protect\mbox{}\protect\hspace*{0pt}\par}
\addcontentsline{toc}{listasb}{\listfigurename}
\pagestyle{especial}
\listoffigures

%:Indice de tablas, coméntese las siguientes líneas si no se desea
\cleardoublepage
\phantomsection
\addcontentsline{toc}{listasb}{\listtablename}
\pagestyle{especial}
\listoftables

%:Indice de Programas
\cleardoublepage
\phantomsection
\addcontentsline{toc}{listasb}{\lstlistlistingname}
\pagestyle{especial}
\lstlistoflistings

%:Bibliografía con biblatex y biber
\cleardoublepage
\phantomsection
\addcontentsline{toc}{listasb}{\bibname}
\pagestyle{especial}
%BIBER
%\printbibliography[heading=etsi]
%BIBTEX
%\bibliographystyle{IEEEtran}
\bibliographystyle{amsplain} %flexbib amsplain alpha
%:Fichero con la bibliografía, BIBTEX
\bibliography{bibliografiaLibroETSI}

%:Índice alfabético de palabras
\cleardoublepage
\phantomsection
\addcontentsline{toc}{listasb}{\indexname}
\chaptermark{\indexname}
\printindex

%:Acrónimos
\cleardoublepage
\phantomsection
\addcontentsline{toc}{listasb}{\glossaryname}
\chaptermark{\glossaryname}
\printglossaries

\end{document}
%